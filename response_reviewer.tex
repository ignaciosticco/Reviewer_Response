\documentclass[a4paper,12pt]{letter}
\usepackage[utf8]{inputenc}
\usepackage{graphicx}
\usepackage{xcolor}
\usepackage{tikz}        %permite rellenar las figuras (circulos)
\usepackage{etoolbox}

\begin{document}
% If you want headings on subsequent pages,
% remove the ``%'' on the next line:
% \pagestyle{headings}

\begin{letter}{Response to the reviewers report}
%\address{Departamento de F\'isica \\ Fac. Ciencias Exactas y Naturales - UBA}

\opening{}

Dear editor, 

The following are our comments on the manuscript ``A re-examination of the role 
of friction in the original Social Force Model''. 



\begin{verbatim}
================================================================
Referee (1)

The paper deals with the well known Helbing's social force model
for simulating pedestrian dynamics.  By  means of some numerical
simulations and comparisons with real data,  the author improves
the model unveiling, at the same time,  the role of the friction
coefficient. The author also shows that the role of the friction
coefficient is basically the same of that of the relaxation time.

The paper is fine and I think it can be published, provided some
corrections are made. More precisely,  I think that the paper is 
not  well  organized  and  the message is difficult to follow. I
strongly  suggest  to  include  Appendix  A  in  the  main  text. 
Section 3  ("Numerical simulations")  should  be named somethink 
like  "Setting  and  parameters".  Section 4 ("Results"),  which 
should   be   the  core  of  the  paper,  is  by  far  too  long. 
Subsection  4.1  ("Hypotheses")  is  fine  but  does not contain 
"results". Subsection 4.2 contains numerical results,  but  they
are not new.

Other comments:

1) Section 1: the  author  could add the reference to two papers
   which introduced the SF idea much before Helbing.

   [A] S. Okazaki, A study of pedestrian   movement   in  archi-
       tectural  space,  Part 1:   pedestrian  movement  by  the 
       application of magnetic model. Trans. A.I.J. 283, 111–119
       (1979).

    [B] K. Hirai,  K. Tarui,  A  simulation  of  the behavior of
        a crowd in panic,  Proceedings   of  the  1975  Interna-
        tional Conference  on  Cybernetics  Society  (1975),  pp.
        409-411.

2) End of page 6: "it is well known that the seminal version..."
   A reference should be added.

3) p.10:  you  cannot  refer  to  $\mathcal A$  and  \mathcal K$ 
   like you do. The  most   important   equations  of Appendix A 
   must  be introduced here.

4) p.14: Fig.1 is not present.

5) Caption Fig.9: k is k_i or k_w?

================================================================
\end{verbatim}

%\newpage

\ps{\textbf{Response to Referee (1):}
\\ 

The following comments are outlined in the same order as appearing in the
reviewer’s report. \\

The reviewer suggests a reorganization of the manuscript. He (she) suggests to include Appendix A in the main text.
We consider that this is a very helpful advice. Therefore, we included Appendix A and Appendix B in the main text (now Sections 5.4 and 5.5 respectively). 
We expect that this modification will strengthen the core message of the paper. \\ 

The reviewer suggests to modify the name of Section 3 (``Numerical Simulations"), to ``Setting and parameters". We followed his (her) suggestion. 

The reviewer points that Section 4 (``Results") is too long and Subsection 4.1 (``Hypotheses")
does not contain results.  We moved the ``Hypotheses" to a new Section previous to Results 
(Section 4 in the revised version is now named ``Hypotheses" and Section 5 corresponds to ``Results"). We also left out Figs. 4 and 5 in order to shorten the Section ``Results". 

The reviewer also made other comments:

1) He (she) mentions some research introducing the SF idea much before 
Helbing \textit{et al.}. We included the corresponding citations 
in the first paragraph of the Introduction. \\


2) The reviewer points out that a specific reference at the end of 
page 6 (``it is well known that the seminal version....") should be included. 
We added the corresponding cite. Note that this text is in page 7 in the revised version and that we have replaced the word ``seminal" by ``original".   \\

% REVISAR ESTE ITEM 
3) In p.10: The reviewer argues that we cannot refer to $\mathcal{A}$ and $\mathcal{K}$ like we did, with no previous introduction. He (she) proposes that the equations of Appendix A can be introduced there. We introduced the most important equations (Eq. 13 and Eq. 14) to give parameters $\mathcal{A}$ and $\mathcal{K}$ a more appropriate presentation. 

4) The reviewer indicates that Fig.1 is not present. Fig.~1 is present in page 8.  \\

We modified the text in order to clarify the fact that the Fig.~1 has a 
dashed circle representing the measurement region for the fundamental diagram. 

5) In Fig. 9, $\kappa$ means $\kappa_i$ and also $\kappa_w$. We replaced $\kappa$ by $\kappa_i = \kappa_w$ to make the caption more clear.   


}

\newpage

\begin{verbatim}
================================================================
Referee (2)

The  paper  studies  the impact of the friction parameter in the 
original SFM.

At first I thought this is a work concerned with the calibration 
of  a  parameter  in an old model for a special case:  corridors. 
This sounds very  specific and not so overwhelmingly interesting,
right? However, I  think this paper contains some nice ideas and 
results  that  are  certainly  interesting  for the community if 
written  and  presented adequately.  In my opinion the very best 
part  of the paper comes in the last 2/3 of the manuscript. Only 
here, the paper starts to show  interesting analysis and results,
I would describe as original.I think the authors would be better
off focusing more on the content in Appendix A and B.  This nice 
normalization, is good. from there work with Eq. (A.3) and go on 
with the clustering analysis.  The  analysis  shown in Fig. 6 is 
also interesting. Here, I wish the authors could shed more light
on the famous boundary effects,  so controversially discussed in
the literature.

Here are some specific comments to the text:

1)  Reading the introduction, I  believe  the  authors know very
    good the  works  done  by Helbing and some of his co-authors,
    especially Johansson. Most of the references in the text are
    citing these two (Ref.2 most of all). I believe, it would be
    good to broaden a bit the literature review in the  introduc-
    tion to more recent works as well.  At last, during the last
    10 years some  positive development in the  community  could 
    be observed, right?

2)  I also suggest  to  reduce  the  obvious  enthusiasm  of the
    authors and dispense with the  use   of superlatives like "a 
    wonderful summary" or "a seminal work" (repeated many times). 
    By the way, Ref. 16 is from 2007.   A  more recent review of
    empirical data can be found here doi : 10.1007/978 - 3 - 642
    - 27737 - 5_706-1"   Empirical  Results  of  Pedestrian  and 
    Evacuation Dynamics"

3)  The very  first paragraph in the paper is not quite accurate
    in my opinion. The force  social force  model ever presented
    was published in K. Hirai and K.Tarui in 1975  (a simulation
    of  the  behavior  of  a  crowd in panic,  Proc. of the 1975 
    International Conference on Cybernetics and  Society. (1975)
    409-411). The model was not called SFM, but it is  a  force-
    model   that   "nicely  bridges  the  socio-psychology  with 
    Newtonian dynamics".

4)  It is not clear to me why the SFM explains why the faster-is
    -slower effect happens. It can be produces, yes, but it does
    not explain why. Please clarify or reformulate.

5)  In  general  the  authors  write  "Helbing  and  co-workers".
    I  suggest  to  use  the  more  formal et al.  This  is also
    more necessary in the references.There sometimes the authors
    use all names of the authors and  sometimes only  the  first
    author followed by et al. Please   also  check  some  errors
    in  some   names  (K\"oster, L\"ohner, . . . )

6)  Equations are missing punctuation.

7)  Section 3: Why is the length of the corridor L=28m while the
    width is w=40m. This sounds strange and  I’m not  sure if it 
    is necessary to have some big values for w.

8)  Section 3: Why are the details about the implementation (C++, 
    LAMMPS) necessary?  I think unless the authors are intending
    to open-source  their  code (which would be nice by the way)
    there is no need to mention these details.  Also the authors
    mention a LAMMPS built in function calculating  the clusters.
    What is this function?

9)  Section 4:  Here  the  authors  cite a  lot  of  other works,
    especially  from  2  and  36, but they do not give their own 
    opinion,  in  regard  of  the  new  findings.  Again  just a 
    reminder Ref 2 is 10 years old. For new findings see Loehner
    PED 2016 in Hefei.

10) Page  10:  vd is  the  desired speed not the "anxiety level".

11) Page 14:  "In  our  case, pedestrians near the walls are the
    ones with the lower velocity".Is this a know empirical fact?
    Why  is  it  so?   Maybe the authors could explain more this
    phenomenon. 

12) I think Fig. 4 and 5 can be safely removed and just replaced 
    with Fig. 6. The normalisation here is nice.

13) The  interesting  phenomenon the authors show in page 23 was
    not  well  analysed  and  explained.  Why  is  it for high k
    pedestrians  stick  together more? Intuitively I would think
    that  high  k  means  high repulsive forces which means that
    pedestrians  stay  away from each other not other way around.

14) In Appendix A the two Parameters in Eq.A.3 are not discussed. 
    Instead the focus is still on tau and k.  How  is this good? 
    Why do you normalize the model, come up with two  parameters
    only  then  to  continue  discussing  the  parameters in the 
    un-normalized model?

To summarize, I think this paper can be published. However, some
heavy  restructuring  and  deeper  analysis  on  the  points  of 
interests may be necessary. 

================================================================

\end{verbatim}


\ps{\textbf{Response to referee (2):} \\

The following comments are outlined in the same order as appearing in the
reviewer’s report. \\

1) The reviewer suggests to include more recent works in the introduction. We consider that this
is a very good advice and we included a paragraph discussing the findings in L\"ohner PED 2016 Hefei.
We also added another paragraph that includes 6 references related to the latest development in the field. \\

2) The reviewer suggests to dispense the use of sentences like ``a wonderful 
summary'' or a ``seminal work''. In order to avoid these expressions, we 
replaced them by more appropriate words throughout the manuscript. \\

On the other side, we acknowledged the citation introduced by the reviewer and 
we included it in the revised manuscript. \\

3) The reviewer points out that the first paragraph of the manuscript is not 
quite accurate. Also, indicates that the Social Force model was introduced much 
before Helbing \textit{et al.} \\

Section 1 (Introduction) has been improved taking into account his (her) 
comments. We further improved the first two paragraphs of the introduction in order to 
make them more accurate (we included the reference that the reviewer suggests). \\ 

4) The reviewer points out that the SFM reproduces the faster-is-slower effect, but the model
does not explain why the faster-is-slower occurs. We agree with the reviewer and reformulated 
the text in order to fulfill this statement.\\

5) The reviewer suggests to change some expressions, like ``co-workers'' by 
the more formal \textit{et al.}. 

We acknowledge this issue and made the corresponding changes in the
revised manuscript. \\

We also corrected the names of the following authors: K\"oster and L\"oner. We defined the same format to all references.   

6) The reviewer noticed that equations are missing punctuation. We added the appropriate punctuation when needed. \\

7) The reviewer wants to know the reasons for the analyzed corridor dimensions. 
\\

The length (28~m) was chosen according to the dimensions of the area analyzed
in Ref. doi.org/10.1103/PhysRevE.75.046109,  corresponding to a specific part 
of the entrance to the Jamaraat Bridge. 
The width was varied upto 40~m just to check out if the velocity profile had a
qualitative difference with the velocity profile corresponding to narrower 
corridors (say 12~m). We found out that regardless the width of the corridor, 
the behavior of the scaled velocity profile remains the same. 

8) The reviewer asks why we mentioned the details of the implementation despite 
we do not open-source our code. \\

In order to allow the reader a better understanding of the simulation process, we 
consider important to point out the details about the implementation. The code is not open-source yet.
Before open-source the code we are working in documentation and clean up to make it easier to understand and maintain by new users.  \\

Furthermore, the reviewer wants to know about the LAMMPS built-in function for 
computing the clusters. \\

We included in Section 3.1 (Simulation software) the name of the LAMMPS 
built-in function and we provided a brief explanation of how this function works. \\ 

9) The reviewer points out that we do not give our own opinion in regard
of the new findings. We included two new paragraphs in the Section Introduction to discuss the new findings of other authors. The revised manuscript contains new references with the contributions of the last 10 years.\\

We prefer to include this discussion in introduction because the results of the paper are more related to the work in Ref.  doi.org/10.1103/PhysRevE.75.046109  \\

10) The reviewer points that vd is the desired speed not the ``anxiety level". 
We acknowledge this issue and made the corresponding change in the manuscript.\\

11) Pedestrians near the walls are the ones with lower velocity because the pedestrians which are closest to the wall are the ones who dissipate more energy due to the friction force (exerted by the wall).\\

The velocity profile resembles the velocity profile measured in\\
Ref.~doi.org/10.1016/j.physa.2013.02.019.
In this work, the authors measure the velocity profile of
pedestrians walking along a straight corridor. We decided to
include this reference in the manuscript to provide empirical 
support to the simulated results. \\

12) The reviewer suggests to remove Fig.~4 and Fig.~5 and just replace them with Fig.~6. We consider that this is a very good advice and we removed Fig.~4 and Fig.~5 since Fig.~6 already contains all the relevant information of the phenomena. \\

13) The reviewer wants to know why as $\kappa$-value increases, more 
pedestrians stick together.

We are glad to explain that $\kappa$ is the coefficient of the 
friction force and, therefore, is not related with the repulsive force 
$\mathbf{f}_s$. Thus, as $\kappa$ increases, more pedestrians stick 
together due the nature of the frictional force. Also, we want to remark that 
the friction force is tangential to the center of mass between two pedestrians, 
while the social force is normal to it. \\

14) The reviewer claims that the two parameters in Eq. A.3 are not discussed.
We acknowledged this issue and added a brief interpretation of the two parameters when introduced.\\

The reviewer asks why we normalize the model and then continue discussing the parameters in the un-normalized model.\\

We prefer to continue discussing the parameters in the un-normalized model to make it more understandable for readers that are used to the un-normalized parameters of the SFM. We are planning to further investigate this normalized parameter and write a new paper focusing on the normalized equation of motion and the role of the different parameters of the SFM. This manuscript is a first attempt to study the model in the framework of the reduced-in-units equation of motion. We pretend to encourage the community to study the SFM (and other models) following this approach.


\begin{verbatim}
================================================================
\end{verbatim}

}

\newpage


\begin{verbatim}
=================================================================
Referee (3)

The authors discussed the role of friction in SFM. The topic  is 
interesting and worthy studying. However, there are issues  that 
should be addressed before processing to a possible  publication 
of this work.

1. Nonstandard reference issues:

 (1)  Numerical  order. For instance, “[10, 12, 8]” (Introduction,
      page 5, paragraph 1).
 (2)  The format of References is inconsistent.

2. The  format  of  unit is nonstandard.  For  instance, “Pm- 2 ”
   (Fig.2) and “p/m 2” (Fig. 3).

3. Numerical simulations, page 8, paragraph 3: The  desired velo-
   city  for  each  pedestrian  was  1 m/s.  Whether the  desired 
   velocity has an effect on the result?

4. Clusters, page 22, paragraph 2: In an enhanced friction scena-
   rio the individuals  find it harder to detach from each other.
   Then why  are  possibilities  of  pedestrian belong to a small 
   cluster and  giant  cluster almost  equal?  Why does a bimodal
   distribution occur in Figs. 9(d)-(f)?

5. Why  does  not the velocity profile report any relevant diffe-
   rence  as  the corridor  widens, while the fundamental diagram
   does?

6. Numerical simulations, page 7, last paragraph: Initially, the 
   individuals were randomly distributed along the corridor, and 
   the corridor with periodic boundary  conditions.  Whether the 
   initial distribute and boundary conditions may have an effect 
   on simulations?

7. English writing should be improved.

================================================================

\end{verbatim}


\ps{\textbf{Response to referee (3):} \\

The following comments are outlined in the same order as appearing in the
reviewer’s report. \\

1-2) The reviewer noticed that the format of references and the format of units is nonstandard. We acknowledge these issues and made the corresponding changes in the manuscript.\\

3) The reviewer asks whether the desired velocity has an effect on the result.\\

The answer to that question is yes. If the desired velocity is much greater than 1~m/s, the flow vs density relation is a monotonic increasing function. If the desired velocity is less than 1~m/s, the results are very similar to 1~m/s. We decided to use 1~m/s because is roughly the average velocity of a moving pedestrian under normal conditions. \\

4) The reviewer asks why are the possibilities of pedestrians belong to a small cluster and giant cluster almost equal. \\

This interesting phenomena occurs because while most of the pedestrians belong to the giant component, some pedestrians remain isolated (``caged" inside the giant component without being in contact with other pedestrians). This produces the bimodal distribution in Figs. 9 (d)-(f).\\

We added a brief comment at the end of the paragraph 2 (Section 5.7) in order to make this concept more clear for the readers.   

5) The reviewer asks why does not the velocity profile report any relevant difference as the corridor widens, while the fundamental diagram does.\\

This happens because the fundamental diagram is strongly dependent on the velocity attain by pedestrians. If the corridor widens, the pedestrians in the middle of the corridor (where the fundamental diagram is measured) reach higher velocities. \\

On the other hand, the qualitative behavior of the velocity profile is the same regardless the size of the corridor (parabolic shape). This does not contradict the fact that the wider the corridor, the higher the maximum velocity reached. The shape of the velocity profile only depends on the boundary conditions in the y-coordinate (fixed walls) that leads to a parabolic velocity profile. \\

We modified the section that shows the velocity profiles. In the revised version we only show the scaled velocity profile. 

6) The reviewer asks whether the initial distribution of pedestrians and boundary conditions may have an effect on the simulations.\\

We checked different initial conditions and saw no significant results. We also tested a situation with a corridor ten times larger (L = 280 m instead of L = 28 m) to reduce the effect of the boundary conditions. We neither noticed any significant change in the result. \\

7) The reviewer suggests to improve the English writing. We agree with the reviewer and modified some sentences to satisfy this important requirement.  \\

\begin{verbatim}
===============================================================
\end{verbatim}

}

Your consideration of this revised manuscript for publication in 
Safety Science is greatly appreciated. 
Please find attached the new version. \\

%\signature{Ignacio Sticco, Guillermo Frank, Fernando Cornes and Claudio Dorso}
\closing{Sincerely}



%enclosure listing
%\encl{}

\end{letter}
\end{document}
